\documentclass[utf8x, 12pt]{G7-32}
\include{preamble-std}

\begin{document}

\frontmatter

\Defines
\begin{description}
        \item[Распределённая система обработки информации] система взаимодействующих
        независимых автоматизированных информационных систем, каждая из которых 
        принадлежит и администрируется различными организациями (лицами), которые 
        преследуют свои собственные цели.
        \item[Участник РСОИ] независимая автоматизированная информационная система,
        входящая в состав распределенной системы обработки информации.  
        \item[Сообщение] минимально передаваемая единица полезной информации.
        \item[Заявка] единица обслуживания.
        \item[Протокол  обмена] это пятёрка:
                \begin{enumerate}
                        \item назначение протокола (предоставляемые им возможности);
                        \item используемый нижестоящий протокол или протоколы;
                        \item алфавит;
                        \item словарь сообщений и синтаксис сообщений;
                        \item возможная последовательность сообщений и их семантика.
                \end{enumerate}
        \item[Жизненный цикл заявки] конечное множество возможных состояний заявки.
        \item[Релевантность] способность информации соответствовать постребностям.
\end{description}

\Abbreviations
\begin{description}
	\item[РСОИ] Распределённая система обработки информации
	\item[АИС] Автоматизированная информационная система
	\item[БД] База данных
	\item[ЛПО] Логика предметной области
	\item[ИП] Интерфейс пользователя
	\item[ПОЗ] Подсистема обработки заявок
	\item[ПФС] Подсистема фильтрации сообщений
	\item[ПОС] Подсистема обмена сообщениями
	\item[ПС] Подсистема статистики
	\item[ПФЗ] Подсистема фильтрации заявок
	\item[ПО] Программное обеспечение
\end{description}

\mainmatter

\chapter{Введение}
Данное техническое задание составлено для проектирования ПО <<Подсистема 
фильтрации нежелательных заявок участника РСОИ>>. Техническое задание выполнено на 
основе ГОСТ 19.201-78. <<ЕСПД. Техническое задание. Требования к содержанию и 
оформлению>>.

\section{Краткое описание предметной области}
Типичная структура РСОИ представлена на рис. ~\ref{fig:rsoi}. АИС, входящие в 
состав РСОИ  являются независимыми и не могут ни получить полностью достоверную 
информацию о состоянии других систем на заданный момент времени, ни польностью 
контролировать её поведение. Каждая из АИС имеет свою логику работы (с точки 
зрения других участников эта логика полностью не просматривается) и свою БД 
или несколько БД, в которых хранит состояние своей модели предметной области. 
Прямой доступ не уровне SQL к этим БД другим участникам не даётся, даже на чтения 
с ограниченными правами. В силу этого системы могут могут выдавать упрощённую или 
искажённую информацию другим участникам РСОИ. В общем случае для создания 
РСОИ всем организациям, владеющим непосредственно взаимодействующими 
учатсниками РСОИ необходимо заключить друг сдругом договор.
\begin{figure}[ht]
        \centering
        \includegraphics[width=0.5\textwidth]{inc/dia/rpz-rsoi}
        \caption{Структура РСОИ, состоящей из 3 независимых АИС}
        \label{fig:rsoi}
\end{figure}

Структура участника РСОИ в общем случае показана на рис. ~\ref{fig:participant}. 
\begin{figure}[ht]
        \centering
        \includegraphics[width=0.7\textwidth]{inc/dia/rpz-participant}
        \caption{Структура участника РСОИ}
        \label{fig:participant}
\end{figure}
ПОС занимается вопросами преобразования полученных от ПОЗ сообщений в форму.
понятную другим системам, и принимаемых от других систем сообщений в форму,
понятную непосредственно самой системе. ПФС фильтрует сообщения во внутреннем 
представлении, чтобы не загружать систему работой, которая выглядит 
неправдоподобной. Входящая в её состав ПФЗ выносит <<вердикт>> по заявкам, 
полученным от других систем, исходя из которого протокол обмена предпринимает те или 
иные дальнейшие шаги. Решение принимается на основе:
\begin{enumerate}
        \item текущего состояния модели предметной области;
        \item истории взаимоотнощений с партнёрами;
        \item параметров заявки (например, количестве запрашиваемых ресурсов);
        \item правил принятия решений.
\end{enumerate}
ПОЗ является ключевой с точки зрения участия в распределенной системе. Она 
реализует часть высокоуровнего простокола участника РСОИ. связанную с жизненным 
циклом выполнения заявки.

\section{Существующие аналоги}
В настоящее время нет аналогичных ПП, позволяющих выполнить анализ релевантности
и соответственно фильтрацию заявок на основе недельно-сезонных колебаний без 
привязки к конкретной предметной области.

\section{Описание подсистемы фильтрации заявок}
	Главное назначение ПО <<Подсистема фильтрации нежелательных заявок участника 
РСОИ>> - определение релевантности поступающих в систему заявок. Алгоритм работы 
подсистемы показан на рис. ~\ref{fig:algorithm}.
\begin{figure}[ht!]
        \centering
        \includegraphics[width=0.9\textwidth]{inc/dia/rpz-algorithm}
        \caption{Алгоритм определения релевантности заявки}
        \label{fig:algorithm}
\end{figure}
Для определения релевантности заявки на основе статистики отношений с клиентом
подсистеме необходимо пройти предварительное обучение с использованием данных,
предоставляемых ей ПС. Правила принятия решений задают дополнительные 
ограничения, накладываемые на поступающие заявки, исходя из здравого смысла и 
бизнес-логики работы системы. Как видно из рисунка правила бывают двух типов.
Выполнение правил 1-го типа проверяется до проверки релевантности системы на
основе статистики отношений с клиентом (например, для данного клиента  
пропускать все поступающие заявки без анализа статистики отношений с ним и т.п.). 
А выполнение правил 2-го типа проводится после него в случае, если заявка была
признана релевантной, исходя из статистики отношений с клиентом (например, заявка на 
количество ресурса, большее половины имеющихся на данный момент в системе, 
признается нерелевантной).

\nobreakingbeforechapters
\chapter{Основания для разработки}

Разработка ведётся в рамках выполнения лабораторных работ по курсу <<Методология
программной инженерии>> на основе утверждённого учебного плана и в рамках курсового
проектирования по курсу <<Рапределённые системы обработки информации>>.

\chapter{Назначение разрабоки}

ПО <<Подсистема фильтрации нежелательных заявок участника РСОИ>>  предназначено
для определения релевантности заявок, поступающих в систему от других учестников
РСОИ, чтобы не загружать систему бесполезной работой.

\chapter{Требования к программному изделию}

\section{Требования к подсистеме}

\begin{enumerate}
        \item Разрабатываемое ПО должно обеспечить функционирование подсистемы
        в режиме 24/7/365.
        \item Время восстановления после сбоя должно составлять не более 1 часа.
\end{enumerate}

\section{Требования к функциональным характеристикам}

\begin{enumerate}
        \item Время переобучения подсистемы должна составлять не более 3-х часов.
        \item Подсистема должна обеспечить обработку не  менее 100 заявок в минуту.
        \item Количество неправильно классифицированных заявок за сутки должно 
        составлять не более 5\% от общей массы заявок, поступающих в систему за этот 
        период.
\end{enumerate}

\section{Требования к реализации}

\begin{enumerate}
        \item Клиенты должны обладать уникальными идентификаторами.
        \item Каждый ресурс системы должен иметь уникальный идентификатор и
        количественную меру его измерения.
        \item Обращение к подсистеме фильтрации заявок для определения релевантности
        заявок должно вестись с использованием протокола XML-RPC.
        \item Для проведения переобучения вызывающая сторона (ПФС) должна обеспечить
        реализацию абстрактного метода доступа к БД статистики отношений с клиентом со
        следующей сигнатурой: get_statistic_records(date, client_id). Входные и выходные 
        параметры метода описаны в таблице ~\ref{tabl:input1}.
        \item Правила принятия решения должны описываться на языке SWI-Prolog. 
        Правила 1 и 2 типа должны быть описаны в разных файлах, пути к которым должны
        задаваться в конфигурационном файле подсистемы. Применение новых правил
        должно быть возможно только после перезапуска подсистемы.
        \item Подсистеме необходимо указывать диапазон дат, статистику за которые
        необходимо использовать для переобучения, в конфигурационном файле.
\end{enumerate}

\begin{table}[ht!]
        \caption{Параметры метода get_statistic_records}
        \begin{center}
                \begin{tabular}{|p{0.3\linewidth}|p{0.3\linewidth}|p{0.2\linewidth}|p{0.2\linewidth}|}
                        \hline
                        \textbf{Название параметра} & \textbf{Назначение} &\textbf{Тип} & \textbf{Допустимые значения}\\
                        \hline
                        client_id & Идентификатор клиента & integer & (0; $2^{32}$) \\
                        \hline   
                        date & Дата, статистика за которую интересует систему & timestamp & (0; $2^{32}$) \\
                        \hline   
                        output & Данные возвращаемые методом & Массив кортежей вида (ResourceID, ResourceCount, ResourceLostCount, OrderState)&  \\
                        \hline   
                        ResourceID & Идентификатор ресурса & integer & (0; $2^{32}$) \\
                        \hline
                        ResourceCount & Запрашиваемое количества ресурса & integer & (0; $2^{32}$) \\
                        \hline
                        ResourceLostCount & Оставшееся количество ресурса & integer & (0; $2^{32}$)\\
                        \hline
                        OrderState & Заключительное состояние заказа & integer & 1 - заказ завершён, 0 - заказ отменён \\
                        \hline   
               \end{tabular}        
               \label{tabl:input3}
        \end{center}
\end{table}

\section{Функциональные требования к подсистеме}

\begin{enumerate}
	\item Подсистема должна обеспечить определение релевантности заявок на основе
        статистики отношений с клиентом и правил принятия решения (1 и 2 типа).
        \item Подсистема должна предоставить возможность переобучения на основе 
        статистики отношений с клиентом.
        \item Подсистема должна предоставить возможность изменения правил принятия 
        решения (1 и 2 типа).
        \item На время переобучения подсистема должна останавливать свою работу.
\end{enumerate}

\subsection{Входные параметры подсистемы}

\begin{enumerate}
        \item Для определения релевантности вход подсистемы должны подаваться данные,
        представленные в таблице ~\ref{tabl:input2}.
        \item Для описания общих правил принятия решений могут использоваться
        переменные, описанные в таблице ~\ref{tabl:input3}.
\end{enumerate}

\begin{table}[ht!]
        \caption{Входные параметры для определения релевантности заявки}
        \begin{center}
                \begin{tabular}{|p{0.4\linewidth}|p{0.4\linewidth}|p{0.2\linewidth}|}
                        \hline
                        \textbf{Название переметра} & \textbf{Тип} & \textbf{Допустимые значения}\\
                        \hline
                        Идентификаторы запрашиваемых ресурсов & Массив, содержащий элементы типа integer & (0; $2^{32}$) \\    
                        \hline   
                        Запрашиваемые количества ресурсов & Массив, содержащий элементы типа integer & (0; $2^{32}$) \\    
                        \hline   
                        Оставшиеся в системе количества запрашиваемых & Массив, содержащий элементы типа integer & (0; $2^{32}$) \\    
                        \hline   
                        Идентификатор клиента & integer & (0; $2^{32}$) \\    
                        \hline   
               \end{tabular}        
               \label{tabl:input2}
        \end{center}
\end{table}

\begin{table}[ht!]
        \caption{Переменные для описания правил принятия решений}
        \begin{center}
                \begin{tabular}{|p{0.3\linewidth}|p{0.3\linewidth}|p{0.2\linewidth}|p{0.2\linewidth}|}
                        \hline
                        \textbf{Название переменной} & \textbf{Назначение} &\textbf{Тип} & \textbf{Допустимые значения}\\
                        \hline
                        ParticipantID & Идентификатор клиента & integer & (0; $2^{32}$) \\
                        \hline   
                        ResourceIDs & Идентификаторы ресурсов & Список, содержащий элементы типа integer & (0; $2^{32}$) \\
                        \hline   
                        ResourceCounts & Количество запрашиваемых ресурсов & Список, содержащий элементы типа integer & (0; $2^{32}$) \\
                        \hline   
                        ResourceLostCounts & Оставшееся количество ресурсов & Список, содержащий элементы типа integer & (0; $2^{32}$) \\
                        \hline   
                        ResourceCount & Размер списка запрашиваемых ресурсов & Список, содержащий элементы типа integer & (0; $2^{32}$) \\
                        \hline   
                        Resource & Идентификатор ресурса & integer & (0; $2^{32}$) \\
                        \hline   
                        Count & Количество ресурса & integer & (0; $2^{32}$) \\
                        \hline   
               \end{tabular}        
               \label{tabl:input3}
        \end{center}
\end{table}

\subsection{Выходные параметры подсистемы}

После определения релеванстности поступившей в систему заявки ПФЗ должна выдать 
решение о целесообразности обработки заявки статистики отношений с клиентом и
правил принятия решений. Значения выходного параметра имеют логический тип и могут 
принимать следующие значения:
\begin{enumerate}
        \item True - завяку целесообразно обработать;
        \item False - иначе.
\end{enumerate}

\section{Требования к надёжности}

Для повышения надёжности работы подсистемы необходимо:
\begin{enumerate}
        \item производить переобучение подсистемы не реже 1 раза в неделю;
        \item производиться журналирование поступающих заявок и принимаемых 
        подсистемой решений;
        \item срок хранения каждого принятого подсистемой решения должен составлять не 
        менее 3 недель.
\end{enumerate}

\section{Требования к составу и параметрам технических средств}

Минимальные технические требования:
\begin{enumerate}
        \item 2-х ядерный процессор с тактовой частотой 2ГГц;
        \item ОЗУ 4 ГБ;
        \item сетевая карта Ethernet стандарта 1000BASE-T.
\end{enumerate}

Требования к операционному окружению:
\begin{enumerate}
        \item Операционные системы: Ubuntu 12.04.
\end{enumerate}

\section{Требования к информационной и программной совместимости}
Разработка должна вестись с использованием открытого платформенно-независимого ПО.

\section{Требования к документации}

Документация должна включать:
\begin{enumerate}
        \item руководство по развертыванию подсистемы;
        \item руководство по настройке подсистемы.
\end{enumerate}

\backmatter

\begin{thebibliography}{00}
        \bibitem{lit1} ГОСТ 19.201-78. ЕСПД. Техническое задание. Требования к содержанию и оформлению.
        \bibitem{lit2} Вишневская Т.И., Романова Т.Н. Технология программирования: Метод. указания к лабораторному практикуму. - Ч. 2. – М: Изд-во МГТУ им. Н.Э. Баумана, 2009
\end{thebibliography}

\end{document}

%%% Local Variables:
%%% mode: latex
%%% TeX-master: t
%%% End:
